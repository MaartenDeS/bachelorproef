\chapter{\IfLanguageName{dutch}{Basisconfiguraties op de servers}{Introduction}}
\label{ch:basisconf}
In dit hoofdstuk wordt voor het eerst een onderzoek uitgevoerd. 

Er wordt voor elke server een configuratie bestand gemaakt dat 4 basisconfiguraties zal uitvoeren: gebruikers en groepen toevoegen, packages installeren en updates, mappen structuur aanmaken, commando's uitvoeren en ssh configuratie. 

Dit zal gebeuren op 1 en meerdere servers voor elke set-up. 

\section{Aanmaken configuratie bestanden}
In dit hoofdstuk zal uitglegd worden hoe de playbooks en cloudconfig bestanden worden aangmemaakt en opgesteld. 

Eerst wordt uitgelegd hoe de cloud-init configuraties worden gedaan, erna het Ansible playbook. Ten laatste zullen de set-ups worden gemaakt waar een combinatie wordt gebruikt.

De gebruiker die zal worden aangemaakt is \textit{bachelor} met wachtwoord \textit{proef}. De gebruiker zal admin zijn en behoren tot de groep \textit{test} die ook zal worden aangemaakt. De packages die zullen worden geïnstalleerd zijn: \textit{pwgen}, \textit{tree} en \textit{git}. De mappenstructuur die zal worden aangemaakt ziet er uit zoals die foto hieronder. Het volgende commando zal worden uitgevoerd: \textit{touch /home/testfile}. En ten laatste zal er op de server een publieke ssh sleutel worden toegevoegd zodat er toegang is vanaf een test server.

insert picture mappen structuur
\subsection{Opstellen cloud-init config bestand}


\subsection{Opstellen Ansible playbook}



1 Server - meerdere servers

Adding User en Groups

Installeren Packages

Running commands

SSH configuratie

MSS OOK SCHIJF CONFIGURATIE + MSS INLEIDING WAT AANPASSEN BIJ HOOFSTUKKEN

OOK OP MEERDERE HOSTS BEKIJKEN

VORIG COMMANDO'S EN BENAMING CHEKCEN

TEST criteria op einde mss aanpassen