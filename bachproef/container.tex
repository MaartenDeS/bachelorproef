\chapter{\IfLanguageName{dutch}{Container \& Cluster Configuratie}{Introduction}}
\label{ch:container}
Dit hoofdstuk voert een laatste onderzoek uit, Containerization.

Hier wordt bekeken hoe Docker het best kan worden geïmplementeerd. Docker is de grootste en populairste technologie in containerization.

\section{Docker}
Docker is een tool dat gemaakt is om het maken, deployen en uitvoeren van applicaties makkelijker te maken door deze in containers te stoppen. 

Containers zijn een package van een applicatie met al zijn dependencies en libraries. Heel de applicatie is nu één enkele package. \autocite{docker}


\subsection{Cloud-init}
Eerst test

\subsection{Ansible}
Eerste test

\subsection{Ansible \& cloud-init}
Eerst test

\section{Uitvoering \& resultaten}

\subsection{Lokaal}

\begin{table}[!htb]
	\centering
	\begin{tabular}{| l | l | l |l |}
        \hline
        \textbf{Uitvoeringstijd} & Resultaat 1 & Resultaat 2 & Resultaat 3   \\ \hline
        cloud-init & 101 sec & 138 sec & 116 sec  \\ \hline
        Anisble & 63.91 sec & 30.16 sec & 33.76 sec \\ \hline
        cloud-init \& Ansible & 116.22 sec & 105.49 sec & 110.47 sec \\
        \hline
    \end{tabular}
	\caption{Resultaten tabel van container configuraties op de servers.}
	\label{tab:tabel resultaten container}
\end{table}





