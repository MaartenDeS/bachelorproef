\chapter{\IfLanguageName{dutch}{Container \& Cluster Configuratie}{Introduction}}
\label{ch:container}
Dit hoofdstuk voert een laatste onderzoek uit, Containerization.

Hier wordt bekeken hoe Docker het best kan worden geïmplementeerd. Docker is de grootste en populairste technologie in containerization.

\section{Docker}
Docker is een tool dat gemaakt is om het maken, deployen en uitvoeren van applicaties makkelijker te maken door deze in containers te stoppen. 

Containers zijn een package van een applicatie met al zijn dependencies en libraries. Heel de applicatie is nu één enkele package. \autocite{docker}

\subsection{Cloud-init}
De configuratie van docker met cloud-init gebeurt via docker-compose. Docker-compose is een manier om docker containers aan te maken te configureren. Er wordt een yaml bestand aangemaakt met de docker-compose modules en verwijzingen. Hierna wordt dit bestand via een commando uitgevoerd.

Het eerste dat wordt gedaan in het cloud-init bestand is installeren van de packages. De package \textit{docker-compose} wordt ook geïnstalleerd. Via deze technologie worden de containers aangemaakt.
\newpage
\begin{lstlisting}[basicstyle=\small]
packages:
- docker.io
- docker-compose
\end{lstlisting}

Als tweede wordt het docker-compose bestand aangemaakt. Dit wordt gedaan via de module \textit{write\_files}. 
\begin{lstlisting}[basicstyle=\small]
write_files:
- path: "home/docker-compose.yml"
content: |

    version: '3.3'
    services:
    db:
    image: mysql:5.7
    restart: always
    environment:
    MYSQL_ROOT_PASSWORD: somewordpress
    MYSQL_DATABASE: wordpress
    MYSQL_USER: wordpress
    MYSQL_PASSWORD: wordpress
\end{lstlisting}

Ten laatste wordt via een commando de container aangemaakt.
\begin{lstlisting}[basicstyle=\small]
runcmd:
-  docker-compose -f home/docker-compose.yml up
\end{lstlisting}

\subsection{Ansible}
Via Ansible kan de docker-container worden aangemaakt met een module in Ansible. Hier wordt niet gewerkt met docker-compose.

Eerst wordt gewoon docker geïnstalleerd.
\begin{lstlisting}[basicstyle=\small]
- name: install docker
become: yes
become_user: root
apt:
pkg:
- docker.io=18.06.*
state: present
update_cache: yes
\end{lstlisting}

Er zijn wel 3 pip packages die moeten worden geïnstalleerd. De eerste is \textit{docker-py} als deze niet werd geïnstalleerd werkte de module \textit{docker\_container} niet. De andere zijn \textit{setuptools} en \textit{wheel}. Dit zijn 2 pip packages die \textit{docker-py} nodig had om te functioneren.
\begin{lstlisting}[basicstyle=\small]
- name: install pip
apt: name=python-pip state=present

- name: pip install setuptools
pip: name=setuptools

- name: pip install wheel
pip: name=wheel

- name: pip install docker-py
pip: name=docker-py
\end{lstlisting}

Als laatste werd de container aangemaakt via de module van Ansible.
\begin{lstlisting}[basicstyle=\small]
- name: Create a mysql container
docker_container:
    name: mysqlcontainer
    image: mysql:5.7
    restart_policy: always
    env:
    MYSQL_ROOT_PASSWORD: somewordpress
    MYSQL_DATABASE: wordpress
    MYSQL_USER: wordpress
MYSQL_PASSWORD: wordpress
\end{lstlisting}

\subsection{Ansible \& cloud-init}
In de omgeving met Ansible en cloud-init werden de packages \textit{docker.io} en \textit{python-pip} via cloud-init geïnstalleerd. De rest werd gedaan via Ansible. Er werd dus weer gewerkt via de Ansible module en niet docker-compose. Ook de pip packages werden dus weer geïnstalleerd.

\section{Uitvoering \& resultaten}
\subsection{Complexiteit}
De complexiteit van beide bestanden is wat hetzelfde. Bij cloud-init is het misschien wat omslachtig om te werken via de docker-compose package, en dit bestand gewoon aan te maken. Al moeten er met Ansible meer taken worden uitgevoerd om een container aan te maken. Alle pip packages moeten worden geïnstalleerd en de python-pip package zelf. 

Ansible kan wel makkelijker worden uitgebreid en aangepast. Er moet gewoon een extra taak worden toegevoegd en er kan een extra docker container worden aangemaakt. Bij cloud-init kan dit ook worden gedaan via het docker-compose bestand. Maar als er meerdere containers blijven toegevoegd worden, gaat dit niet meer overzichtelijk zijn. Bij Ansible kunnen alle aparte containers nog goed onderscheiden worden. Ook is de ouput bij Ansible hiervoor veel handiger. Er kan perfect worden meegevolg welke container al aangemaakt is, en welke nog niet.

\subsection{Snelheid}
Als we naar snelheden (Tabel \ref{tab:tabel resultaten container}) kijken is Ansible  weer de beste. Zelfs met de aftrekking van de initïele startconfiguraties komen de andere 2 niet in de buurt. Dit is weer een duidelijk pluspunt voor Ansible.
\begin{table}[!htb]
	\centering
	\begin{tabular}{| l | l | l |l |}
        \hline
        \textbf{Uitvoeringstijd} & Resultaat 1 & Resultaat 2 & Resultaat 3   \\ \hline
        cloud-init & 101 sec & 138 sec & 116 sec  \\ \hline
        Anisble & 63.91 sec & 30.16 sec & 33.76 sec \\ \hline
        cloud-init \& Ansible & 116.22 sec & 105.49 sec & 110.47 sec \\
        \hline
    \end{tabular}
	\caption{Snelheid tabel van container configuraties op de servers.}
	\label{tab:tabel resultaten container}
\end{table}

\subsection{Resultaat}
Voor containerization lijkt het in eerste instantie beter om voor Ansible te gaan. Het uitbreiden naar meerdere containers is het makkelijkst met Ansible, wat toch de doorslag geeft. De snelheid is ook een pluspunt.

Als er een docker server wordt opgezet met maar 1 container (of een klein aantal) kan er wel overwogen worden om voor cloud-init te gaan. Zeker als dit moeten worden opgezet bij de startup van de server. Want het meegeven van een cloud-init script, is nog steeds veel makkelijker dan het meegeven van een Ansible script. 

Als er docker server moet worden opgezet met meerdere containers bij de start up, is het best om cloud-init en Ansible te gebruiken. Weer door het makkelijk meegeven van het cloud-init script. Hierin kan er dan ook makkelijker worden verwezen naar het Ansible playbook.
