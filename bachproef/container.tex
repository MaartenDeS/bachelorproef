\chapter{\IfLanguageName{dutch}{Container \& Cluster Configuratie}{Introduction}}
\label{ch:container}
Dit hoofdstuk voert een laatste onderzoek uit. Containerization en clusterization is niet meer weg te denken uit systeembeheer. 

Hier wordt bekeken hoe Docker en Kubernetes het best kan worden geïmplementeerd. Docker en Kubernetes zijn beide respectievelijke de grootste en populairste technologiëen in containerization en clusterization.

\section{Docker}
Docker is een tool dat gemaakt is om het maken, deployen en uitvoeren van applicaties makkelijker te maken door deze in containers te stoppen. 

Containers zijn een package van een applicatie met al zijn dependencies en libraries. Heel de applicatie is nu één enkele package. \autocite{docker}


\subsection{Cloud-init}
Eerst test

\subsection{Ansible}
Eerste test

\subsection{Ansible \& cloud-init}
Eerst test

\section{Kubernetes}
Kubernetes is een platform voor het beheren van containerizede werklast en services. Het vergemakkelijkt de configuratie en automatisering.

\subsection{Cloud-init}
Eerste test

\subsection{Ansible}
Eerste test

\subsection{Ansible \& cloud-init}
Eerst test

\section{Resultaten}

\subsection{Lokaal}

\subsection{Cloud}