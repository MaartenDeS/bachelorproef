%%=============================================================================
%% Inleiding
%%=============================================================================

\chapter{\IfLanguageName{dutch}{Inleiding}{Introduction}}
\label{ch:inleiding}
In dit hoofdstuk wordt er een kort inleiding over de bachelorproef. De oorsprong van het idee en de onderzoeksvraag wordt besproken. Ook wordt er al wat basisinfo gegeven over het onderwerp.


De installatie en modificatie van software servers moet voor de gebruiker altijd makkelijker en sneller. Eens de gebruiker weet wat voor server hij wil, wil hij deze liefst zo snel mogelijk opzetten met de nodige specificaties. Of als de gebruiker een kleine aanpassing wil doen aan de server, wil hij dit zo makkelijk mogelijk aanpassen. 

Om dit zo efficiënt mogelijk te doen heb je configuration management tools nodig. Dit zijn tools die gemaakt zijn om software op servers te installeren en te beheren. De meest bekende tools zijn Chef, Puppet, Salt en Ansible. In deze bachelorproef gaat het onder andere over Ansible. 

%De inleiding moet de lezer net genoeg informatie verschaffen om het onderwerp te begrijpen en in te zien waarom de onderzoeksvraag de moeite waard is om te onderzoeken. In de inleiding ga je literatuurverwijzingen beperken, zodat de tekst vlot leesbaar blijft. Je kan de inleiding verder onderverdelen in secties als dit de tekst verduidelijkt. Zaken die aan bod kunnen komen in de inleiding~\autocite{Pollefliet2011}:

%\begin{itemize}
%  \item context, achtergrond
%  \item afbakenen van het onderwerp
 % \item verantwoording van het onderwerp, methodologie
 % \item probleemstelling
  %\item onderzoeksdoelstelling
  %\item onderzoeksvraag
  %\item \ldots
%\end{itemize}

\section{\IfLanguageName{dutch}{Context}{Problem Statement}}
\label{sec:context}

Ansible is op zich al een zeer goede configuration management tool, maar voor het bedrijf Be-Mobile nog niet goed genoeg. 

Be-Mobile is een Big Data verkeersbedrijf. Be-Mobile wil verkeer revolutioneren en de mobiliteits oplossingen voor morgen en vandaag creëren. Hun hoofdzetel ligt in Melle, bij Gent. 

Be-Mobile werkt met "Hetzner Cloud". Dit is hun cloud server provider. Met hetzner cloud heb je de optie om te verwijzen naar een cloudconfig file om je server te configureren. Het cloudconfig bestand is het configuratie  bestand van cloud-init. 

Cloud-init is net zoals een Ansible een soort van configuration management tool maar speciaal voor cloud servers. Door middel van paramaters in te vullen in dit bestand kan je jouw server configureren.
 

%Uit je probleemstelling moet duidelijk zijn dat je onderzoek een meerwaarde heeft voor een concrete doelgroep. De doelgroep moet goed gedefinieerd en afgelijnd zijn. Doelgroepen als ``bedrijven,'' ``KMO's,'' systeembeheerders, enz.~zijn nog te vaag. Als je een lijstje kan maken van de personen/organisaties die een meerwaarde zullen vinden in deze bachelorproef (dit is eigenlijk je steekproefkader), dan is dat een indicatie dat de doelgroep goed gedefinieerd is. Dit kan een enkel bedrijf zijn of zelfs één persoon (je co-promotor/opdrachtgever).

\section{\IfLanguageName{dutch}{Probleemstelling - Onderzoeksvraag}{Research question}}
\label{sec:probleemstellingonderzoeksvraag}

Het probleem is meteen heel duidelijk. Moet er worden overgestapt naar cloud-init in plaats van Ansible. In deze bachelorproef gaat dit worden onderzocht. Waar zijn Ansible en cloud-init verschillend, waar zijn ze hetzelfde en waar vullen ze mekaar aan. De echte onderzoeksvragen waar deze thesis een antwoord op hoopt te vinden is:

\begin{itemize}
    \item Is Ansible overbodig door het gebruikt van Cloud-init?
    \item Zijn Ansible en cloud-init compatibel?
    \item Op welke manier zijn ze compatibel of overbodig?
\end{itemize}



%Wees zo concreet mogelijk bij het formuleren van je onderzoeksvraag. Een onderzoeksvraag is trouwens iets waar nog niemand op dit moment een antwoord heeft (voor zover je kan nagaan). Het opzoeken van bestaande informatie (bv. ``welke tools bestaan er voor deze toepassing?'') is dus geen onderzoeksvraag. Je kan de onderzoeksvraag verder specifiëren in deelvragen. Bv.~als je onderzoek gaat over performantiemetingen, dan 

%\section{\IfLanguageName{dutch}{Onderzoeksdoelstelling}{Research objective}}
%\label{sec:onderzoeksdoelstelling}
%misschien doen
%Wat is het beoogde resultaat van je bachelorproef? Wat zijn de criteria voor succes? Beschrijf die zo concreet mogelijk. Gaat het bv. om een proof-of-concept, een prototype, een verslag met aanbevelingen, een vergelijkende studie, enz.

\section{\IfLanguageName{dutch}{Opzet van deze bachelorproef}{Structure of this bachelor thesis}}
\label{sec:opzet-bachelorproef}

% Het is gebruikelijk aan het einde van de inleiding een overzicht te
% geven van de opbouw van de rest van de tekst. Deze sectie bevat al een aanzet
% die je kan aanvullen/aanpassen in functie van je eigen tekst.

De rest van deze bachelorproef is als volgt opgebouwd:

In Hoofdstuk~\ref{ch:inleidingtotansibleencloudinit} wordt een inleiding gegeven tot Ansible en cloud-init

In Hoofdstuk~\ref{ch:stand-van-zaken} wordt een overzicht gegeven van de stand van zaken binnen het onderzoeksdomein, op basis van een literatuurstudie.

In Hoofdstuk~\ref{ch:methodologie} wordt de methodologie toegelicht en worden de gebruikte onderzoekstechnieken besproken om een antwoord te kunnen formuleren op de onderzoeksvragen.

In Hoofdstuk~\ref{ch:testlokaal} worden lokale testomgevingen  opgezet doormiddel van Vagrant en Virtualbox.

In Hoofdstuk~\ref{ch:testhetzner} worden testomgevingen opgezet doormiddel van Hetzner Cloud.

In Hoofdstuk~\ref{ch:basisconf} worden er basisconfiguraties gedaan op alle servers (aanmaken users, package installeren,..)

In Hoofdstuk~\ref{ch:serverconf} worden er verschillende soorten servers(file, web, dhcp, ...) gemaakt met de testomgevingen.

In Hoofdstuk~\ref{ch:naopstarten} wordt er gekeken hoe je instellingen aanpast na het opstarten van de server.

In Hoofdstuk~\ref{ch:container} wordt gekeken de configuratie containers en clusters wordt ondersteunt.

In Hoofdstuk~\ref{ch:conclusie}, tenslotte, wordt de conclusie gegeven en een antwoord geformuleerd op de onderzoeksvragen. Daarbij wordt ook een aanzet gegeven voor toekomstig onderzoek binnen dit domein.