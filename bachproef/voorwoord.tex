%%=============================================================================
%% Voorwoord
%%=============================================================================

\chapter*{\IfLanguageName{dutch}{Woord vooraf}{Preface}}
\label{ch:voorwoord}

%% TODO:
%% Het voorwoord is het enige deel van de bachelorproef waar je vanuit je
%% eigen standpunt (``ik-vorm'') mag schrijven. Je kan hier bv. motiveren
%% waarom jij het onderwerp wil bespreken.
%% Vergeet ook niet te bedanken wie je geholpen/gesteund/... heeft

Deze bachelorproef rondt mijn opleiding Toegepaste Informatica aan de HoGent af. 

Als onderwerp voor deze bachelorproef heb ik besloten om 2 technologieën, Ansible en cloud-init, met elkaar te vergelijken en na te gaan of een samenwerking mogelijk is. Het bedrijf Be-Mobile heeft me op dit idee gebracht. Zij hadden gehoord van een nieuwe technologie cloud-init en vroegen zich af of dit hun huidige tool, Ansible, kon vervangen en, of dat deze konden samenwerken. Ik vond dit een zeer interessant onderwerp, aangezien ik al bekend was met Ansible. Na een klein onderzoek naar cloud-init heb ik besloten om hierover mijn bachelorproef te maken.

Deze bachelorproef is een werk waar ik persoonlijk wel trots op ben. Maar dit was niet mogelijk geweest zonder de hulp van een aantal mensen. Graag zou ik ze hier even willen bedanken. 

Eerst en vooral zou ik graag het bedrijf Be-Mobile en in het bijzondere mijn co-promoter Simon Lepla willen bedanken. Ze gaven mij de nodige middelen om de bachelorproef goed uit te voeren. Bij vragen over technische zaken kon ik ook atlijd bij hen terecht. Graag zou ik ook mijn promoter Lotte Van Steenberghe van de HoGent willen bedanken. Ze heeft mij goed begeleid tijdens deze periode. Ook zou ik graag mijn familie willen bedanken voor hun motivatie en in het bijzonder mijn moeder. Zij heeft tijd vrijgemaakt om mijn bachelorproef na te lezen op eventuele spelfouten.

Ik hoop dat de inhoud van mijn bachelerproef u iets kan kan bijleren!
\begin{flushright}
    \textit{Maarten De Smedt,} \\ 
    \textit{Academiejaar 2018-2019}
\end{flushright}
