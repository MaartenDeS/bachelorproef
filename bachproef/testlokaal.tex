\chapter{\IfLanguageName{dutch}{Opzetten lokale testomgeving}{Introduction}}
\label{ch:testlokaal}

In dit hoofstuk wordt besproken hoe de lokale testomgevingen zijn opgezet. Er zijn 3 lokale testomgevingen. De eerste is één die cloud-init gebruikt, de tweede gebruikt Ansible en de derde gebruikt cloud-init en Anisble. Voor het opzetten wordt er VirtualBox en Vagrant gebruikt.

\section{Cloud-init}
De eerste testomgeving die wordt besproken is die met cloud-init. Hij werd opgezet met behulp van: \autocite{cloudVagrant}. Er wordt hiervoor in 2 stappen gewerkt. Eerst wordt er een ISO met de cloud-init configuraties aangemaakt met behulp van de setup van \autocite{cloudVagrant}. Erna wordt deze ISO toegevoegd aan de testomgeving.

\subsection{Maken van ISO}
Voor het aanmaken van de ISO is er een Linux omgeving nodig. Allereest wordt er een omgeving opgezet om de ISO's in te maken. Deze wordt ook opgezet met Vagrant en VirtualBox. De box \textit{ubuntu/xenial64} wordt gebruikt (dit is dezelfde als degene die later in de testomgeving wordt gebruikt). Met het commando \textit{vagrant up} wordt de server aangemaakt. De vagrantfile ziet er zo uit:
\begin{lstlisting}
Vagrant.configure(''2'') do |config|
	config.vm.box =''ubuntu/xenial64''
end
\end{lstlisting}

Via het commando \textit{vagrant ssh} hebben we de toegang tot de server. Eens die er is tot de machine, wordt de package \textit{genisoimage} geïnstalleerd. Dit wordt via het commando \textit{sudo apt-get install genisoimage}. 

Nadat die package geïnstalleerd is wordt er een user-data bestand aangemaakt. In dit bestand komt de cloudconfig data. Hierna wordt een tweede bestand aangemaakt, namelijk het meta-data bestand. In dit bestand komt er nog extra minimale configuratie. De local-hostname en instance-id moeten hierin worden gezet.

Als deze bestanden zijn aangemaakt wordt er een Makefile gecreeërd. Via dit bestand wordt de iso aangmaakt. Zo ziet de Makefile eruit:
\begin{lstlisting}
nocloud.iso: meta-data user-data
    mkisofs \
        -joliet -rock \
        -volid "cidata" \
        -output nocloud.iso meta-data user-data
\end{lstlisting}

Ten laatste wordt het \textit{make} package geïnstalleerd. Via het commando \textit{make} wordt het iso bestand aangemaakt.

\subsection{Opzetten van testomgeving met cloud-init configuraties}
Er wordt een tweede omgeving aangemaakt (de effectieve testomeving) weer doormiddel van Vagrant en VirtualBox. Net zoals in het vorige gedeelte wordt ook hier de box \textit{ubuntu/xenial64} gebruikt.

In het vorige gedeelte is het iso bestand aangemaakt met de cloud-init configuraties. Allereerst moet dit iso bestand worden gekopieerd naar de map van de omgeving. In de vagrantfile die wordt gebruikt moet er dan worden doorverwezen deze iso. Dit wordt gedaan met deze configuraties in de vagrantfile.
\begin{lstlisting}
IMAGE_PATH = File.join(File.dirname(__FILE__), "no-cloud.iso")

Vagrant.configure(2) do |config|
	
    config.vm.box = "ubuntu/xenial64"
	
    config.ssh.password = nil	
    config.vm.synced_folder ".", "/vagrant", disabled: true
	
    config.vm.provider :virtualbox do |vb|
	  vb.customize [
	    "storageattach", :id,
	    "--storagectl", "SCSI",
	    "--port", "1",
	    "--type", "dvddrive",
	    "--medium", IMAGE_PATH
	]
	vb.linked_clone = true
   end
end
\end{lstlisting}
Als hierna \textit{vagrant up} wordt gedaan, wordt de server aangemaakt met de configuraties van cloud-init.
 
\section{Ansible}
De tweede lokale testomgeving die wordt opgezet is die met Ansible. Deze is niet zo complex om op te zetten als degene met cloud-init. Deze wordt opgezet met behulp van \autocite{ansibleVagrant}. Ook deze omgeving gebruikt Vagrant en VirtualBox. Als box wordt ook weer dezelfde gebruikt als de vorige keren. 

Er zijn eigenlijk maar 2 stappen die moeten worden gedaan. Allereest moet er een playbook bestand worden aangemaakt. Deze moet ook aanwezig in de locatie van de omgeving. De tweede stap is verwijzen naar dit playbook via de vagrantfile. Dit wordt op deze manier gedaan.
\begin{lstlisting}
Vagrant.configure(2) do |config|
    config.vm.box = "ubuntu/xenial64"
    
    config.vm.provision "ansible" do |ansible|
      ansible.verbose = "v"
      ansible.playbook = "playbook.yml"
   end
end
\end{lstlisting}

\section{Cloud-init \& Ansible }
to do