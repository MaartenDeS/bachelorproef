\chapter{\IfLanguageName{dutch}{Literatuurstudie}{State of the art}}
\label{ch:stand-van-zaken}


% Tip: Begin elk hoofdstuk met een paragraaf inleiding die beschrijft hoe
% dit hoofdstuk past binnen het geheel van de bachelorproef. Geef in het
% bijzonder aan wat de link is met het vorige en volgende hoofdstuk.

% Pas na deze inleidende paragraaf komt de eerste sectiehoofding.

%Dit hoofdstuk bevat je literatuurstudie. De inhoud gaat verder op de inleiding, maar zal het onderwerp van de bachelorproef *diepgaand* uitspitten. De bedoeling is dat de lezer na lezing van dit hoofdstuk helemaal op de hoogte is van de huidige stand van zaken (state-of-the-art) in het onderzoeksdomein. Iemand die niet vertrouwd is met het onderwerp, weet nu voldoende om de rest van het verhaal te kunnen volgen, zonder dat die er nog andere informatie moet over opzoeken \autocite{Pollefliet2011}.

%Je verwijst bij elke bewering die je doet, vakterm die je introduceert, enz. naar je bronnen. In \LaTeX{} kan dat met het commando \texttt{$\backslash${textcite\{\}}} of \texttt{$\backslash${autocite\{\}}}. Als argument van het commando geef je de ``sleutel'' van een ``record'' in een bibliografische databank in het Bib\LaTeX{}-formaat (een tekstbestand). Als je expliciet naar de auteur verwijst in de zin, gebruik je \texttt{$\backslash${}textcite\{\}}.
%Soms wil je de auteur niet expliciet vernoemen, dan gebruik je \texttt{$\backslash${}autocite\{\}}. In de volgende paragraaf een voorbeeld van elk.

%\textcite{Knuth1998} schreef een van de standaardwerken over sorteer- en zoekalgoritmen. Experten zijn het erover eens dat cloud computing een interessante opportuniteit vormen, zowel voor gebruikers als voor dienstverleners op vlak van informatietechnologie~\autocite{Creeger2009}.

In dit hoofdstuk wordt een literatuur studie gedaan. Er worden 2 artikels gevonden die worden besproken: 

\textbf{An introduction to server provisioning with CloudInit} en \textbf{Using Ansible to Bootstrap My Work Environment Part 4}. Deze waren beiden ook een onderdeel van het bachelorproef voorstel.

\section{An introduction to server provisioning with CloudInit}
Het eerste artikel dat gevonden werd is: \textbf{An introduction to server provisioning with CloudInit}. Het is geschreven door Viktor Petersson. \autocite{viktorpet} beschrijft in zijn artikel de basis van cloud-init en hoe een er mee kan worden gewerkt op CloudSigma. 

In deze literatuurstudie wordt er besproken wat de link met Ansible is volgens \autocite{viktorpet}. Een ander groot onderdeel van het artikel is hoe het Cloud config bestand wordt opgemaakt. En wat dit bestand heeft/geeft qua voordelen. Ook is er een deel waar er wordt getoond hoe er een CloudSigma server wordt opgesteld. Dit wordt niet besproken. Dit is een klein deel van het artikel en niet van nut voor deze bachelorproef/onderzoek.

\newpage
\subsection{Verschil en samenwerking met Ansible volgens Viktor Petersson}
\autocite{viktorpet} wil allereerst de lezer doen inzien dat cloud-init een specifieke plaats heeft in de server provisioning wereld. In \autocite{viktorpet} wordt er vermeld dat Cloud-init namelijk een bepaalde eigenschap heeft, die vele provisioning systemen zoals een Ansible, Puppet of Chef niet hebben. 

Hoewel cloud-init perfect kan gebruikt worden als een stand-alone provisioning systeem. Is het één van de weinige systemen die het gebruik met andere provisioning systeem ondersteunt, en zelf aanraadt volgens \autocite{viktorpet}. \autocite{viktorpet} vermeldt ook dat hij prefereert cloud-init te gebruiken boven Ansible.

\subsection{Setup cloud config bestand}
Het volgende onderdeel van \autocite{viktorpet} dat we bespreken is het opstellen van zijn cloud config bestand. Dat is het bestand waar de server configuraties wordt ingevoerd. \autocite{viktorpet} legt aan de hand van verschillende functies van het cloud config bestand uit wat de voordelen van cloud-init zijn. In Hoofdstuk \ref*{ch:cloud-init} is dit al meer besproken. Sommige delen gaan minder uitgebreid zijn omdat het al in Hoofdstuk \ref*{ch:cloudmodules} wordt uitgelegd.

\subsubsection{SSH}
Het eerste onderdeel van het cloud config bestand van \autocite{viktorpet} zijn de SSH keys. Cloud-init is volgens \autocite{viktorpet} handig om ssh keys toe te voegen aan de server. Zo kan er verbinding gemaakt worden met de server. 

\autocite{viktorpet} gebruikt 2 fictieve gebruikers en SSH Keys in zijn voorbeeld. Deze publieke SSH sleutels worden geïnstalleerd op de server voor de gekozen gebruiker. In het fictieve voorbeeld dus een een publieke sleutel voor user1 en user2 op de server host. Op Ubuntu Cloud Images als de gebruikers niet aanwezig zijn, zullen deze worden geïnstalleerd op de standaard gebruiker ubuntu. 

\autocite{viktorpet} maakt verbinding met de server via het commando \textit{ssh ubuntu@IPADRESS}. Via de toegevoegde public keys gaat dit zonder problemen.
\begin{lstlisting}
#cloud-config
ssh_authorized_keys:
 - ssh-rsa AAA...user1@host
 - ssh-rsa AAA...user2@host
\end{lstlisting} 

\newpage
\subsubsection{Systeem updates}
Het volgende voordeel van cloud-init volgens \autocite{viktorpet} is het uitvoeren van systeem updates tijdens de eerste boot. Dit is een van de functies die al werd besproken in Hoofdstuk \ref*{ch:cloudmodules}.

In zijn voorbeeld gebruikt \autocite{viktorpet} \textit{apt\_updgrade}. Terwijl er in Hoofstuk \ref*{ch:cloudmodules} werd gezien dat updates worden gedaan met \textit{package\_upgrade}. \textit{apt\_upgrade} is gewoon een alias voor \textit{package\_upgrade}.

Ook is \textit{apt\_update} zijn standaard waarde al true, als er packages worden geïnstalleerd op de server. 
\begin{lstlisting}
#cloud-config
apt_update: true
apt_upgrade: true
\end{lstlisting} 

\subsubsection{Installeren packages}
Het volgende dat wordt bespreken is het installeren van packages. Ook dit werd al besproken in Hoofdstuk \ref*{ch:cloudmodules}. Extra informatie die nog niet gekend was geeft \autocite{viktorpet} niet. Zijn voorbeeld wordt hieronder vermeld.
\begin{lstlisting}
#cloud-config
packages:
 - python-pip
 - fail2ban
 - vim
\end{lstlisting} 

\subsubsection{Hostname}
Wat \autocite{viktorpet} ook handig vindt aan cloud-init is het aanpassen van de hostname. Deze kan ook makkelijk worden aangepast.
\begin{lstlisting}
#cloud-config
hostname: mynode
fqdn: mynode.example.com
manage_etc_hosts: true
\end{lstlisting} 

\subsubsection{Commando's}
Als er sprake is van een meer geavanceerde gebruiker moeten er ook commando's worden uitgevoerd. Er zijn 2 opties om commando's uit te voeren \textit{runcmd} en \textit{bootcmd}. Hierover kan er ook meer informatie worden gevonden in Hoofdstuk \ref*{ch:cloudmodules}.
\begin{lstlisting}
#cloud-config
runcmd:
 - ls -l /root
\end{lstlisting} 
\subsubsection{Server coniguration manager}
Een laatste voordeel volgens \autocite{viktorpet} is dat als meer geavanceerde gebruiker er ook de mogelijkheid is om een extra Server Configuaration Manager te gebruiken. 

Cloud-init ondersteunt samenwerking met onder andere Chef, Puppet en Salt.
\subsubsection{Alles samenbinden}
Ten laatste toont \autocite{viktorpet} hoe dit allemaal kan worden samengevoegd in een bestand. Dit wordt allemaal samengevoegd in een YAMl bestand en bovenaan wordt er \textit{\#cloud-config} gezet. Dit werd ook vermeld in Hoofdstuk \ref*{ch:cloud-init}
\begin{lstlisting}
#cloud-config
ssh_authorized_keys:
- ssh-rsa AAA...user1@host
- ssh-rsa AAA...user2@host

hostname: mynode
fqdn: mynode.example.com
manage_etc_hosts: true

apt_update: true
apt_upgrade: true

packages:
- python-pip
- fail2ban
- vim

runcmd:
- ls -l /root
\end{lstlisting} 

\newpage

\section{Using Ansible to Bootstrap My Work Environment Part 4}
Het tweede artikel is: \textbf{Using Ansible to Bootstrap My Work Environment Part 4}. Het is een Blogpost geschreven door Scott Harney. \autocite{scottharney} beschrijft in zijn blogpost hoe hij zijn werk omgeving opstart via Ansible met behulp van cloud-init. 

% tot hier was spellcheck en woorden bekeken.

Het is het beste artikel dat werd gevonden waar Ansible en cloud-init beide expliciet werden gebruikt. Er werd voor het eerste een omgeving weergegeven die opgezet werd door een samenwerking van beide. In \autocite{scottharney} wordt beschreven hoe Scott Harney een EC2 instance op zet op AWS. AWS is 1 van de cloud providers die cloud-init ondersteunt. Het artikel is opgedeeld in 4 onderdelen. 

\textit{'Initial provisioning'}, in dit deel wordt de EC2 instance opgestart. Dit gebeurt met een eerste Ansible playbook dat ondere ook verwijst naar het cloud-init script dat nodig is. \textit{'Cloud-init'}, in het 2de gedeelte wordt het cloud-init script opgemaakt en uitgelegd wat alles doet en betekent. \textit{'Ansible folow up playbook'}, in dit gedeelte wordt de EC2 instance geconfigureerd met Ansible. \textit{'Launch and configure'} hier wordt de omgeving opgestart en worden sommige variablen nog ingevuld.

Het literatuur onderzoek gaat over de eerste 3 onderdelen gaan: initial provisioning, cloud-init en Ansible follow up playbook. Hierin worden Ansible en cloud-init besproken en geconfigureerd.

\subsection{Ansible initial playbook}
Er werd gekozen om de initiële provisioning en post-provisioning in 2 verschillende playbooks te doen. \autocite{scottharney} vond dit het handigste en vind het onnodig om deze tot 1 playbook te maken. 

Dit playbook zorgt voor het aanmaken en opstarten van de instance. Dit wordt gedaan via de Ansible role \textit{ec2}. Via deze role en dit playbook wordt ook het cloud config bestand meegegeven. \autocite{scottharney} gaf dit mee door middel van de optie \textit{user\_data}. 

\subsection{Cloud-init}
\autocite{scottharney} beschreef hier hoe hij de het cloud config bestand, waar hij in het eerste playbook naar verwees, heeft geconfigureerd. Een eerste opmerking van \autocite{scottharney} was dat cloud-init niet veel goede documentatie had. \autocite{scottharney} vond dit opmerkelijk omdat het een populaire tool is. Vervolgens beschreef \autocite{scottharney} hoe zijn cloud config bestand is opgedeeld.

\newpage
\subsubsection{Begin}
Allereerst werd de normale gebruiker van \autocite{scottharney} aangemaakt met de nodige configuraties.
\begin{lstlisting}
#cloud-config
users:
- name: {{ ansible_user }}
ssh-authorized-keys:
- ssh-rsa  umm. nope
groups: [ 'admin', 'adm', 'dialout', 'sudo', 'lxd', 
		'plugdev', 'netdev' ]
shell: /bin/bash
sudo: ["ALL=(ALL) NOPASSWD:ALL"]
\end{lstlisting} 

\subsubsection{Server en host update}
In het 2de deel van het bestand nam \autocite{scottharney} zijn Hostname en fqdn en werden de packages upgedate. Ook update hij de instance zijn \textit{/etc/host} en update hij de tijdzone.
\begin{lstlisting}
hostname: "{{ item.hostname }}"
fqdn: "{{ item.fqdn }}"
manage_etc_hosts: true
timezone: US/Central
package_update: true
package_upgrade: true
\end{lstlisting} 

\subsubsection{Packages}
Om het laatste van zijn script uit te voeren had \autocite{scottharney} 2 packages nodig, namelijk: \textit{python} en \textit{awscli}. 
\begin{lstlisting}
packages:
- awscli
- python
\end{lstlisting}

\newpage
\subsubsection{DNS zone bestand}
Het laatste deel was het belangrijkste deel van het cloud config bestand. \autocite{scottharney} moest zijn DNS zone bestand updaten. Zodat zijn EC2 instance correct werkte. Dit deed hij door met de optie \textit{write\_files} een script aan te maken om deze te updaten. Hij prefereerde deze methode dan het the pushen via \textit{systemd}.
\begin{lstlisting}
write_files:
- content: |
#!/bin/sh
FQDN=`hostname -f`
ZONE_ID="{{ zone_id }}"
TTL=300
SELF_META_URL="http://169.254.169.254/latest/meta-data"
PUBLIC_DNS=$(curl ${SELF_META_URL}/public-hostname 2>/dev/null)

cat << EOT > /tmp/aws_r53_batch.json
{
"Comment": "Assign AWS Public DNS as a CNAME of hostname",
"Changes": [
{
"Action": "UPSERT",
"ResourceRecordSet": {
"Name": "${FQDN}.",
"Type": "CNAME",
"TTL": ${TTL},
"ResourceRecords": [
{
"Value": "${PUBLIC_DNS}"
}]}}]}
EOT

aws route53 change-resource-record-sets --hosted-zone-id ${ZONE_ID} 
		--change-batch file:///tmp/aws_r53_batch.json
rm -f /tmp/aws_r53_batch.json
path: /var/lib/cloud/scripts/per-boot/set_route53_dns.sh
permissions: 0755e
\end{lstlisting}
\newpage
\subsection{Ansible follow up playbook}
Nadat de instance is opgestart en de eerste configuraties zijn uitgevoerd. Is het tijd voor het tweede playbook. Het eerste deel van het playbook ziet er zo uit.
\begin{lstlisting}
- hosts: tag_Name_candyapplegrey
  gather_facts: True
  roles:
    - role: ec2
    - role: common
    - role: ansible_mystuff
    - role: openvpn_server

  tasks:
  - name: Reboot system if required
    tags: reboot
    become: yes
    command: /sbin/reboot  removes=/var/run/reboot-required 
    async: 1
    poll: 0
    ignore_errors: true

- name: waiting for {{ inventory_hostname }} to reboot
  local_action: wait_for host={{ inventory_hostname }} state=
                                started delay=30 timeout=300 
  become: no
\end{lstlisting}

\subsubsection{EC2 rol}
Dit deel van het playbook voegde de nieuwe publieke sleutel van de opgestarte host toe aan het \textit{known hosts} bestand. Maar Ansible zal ook vragen de sleutel te accepteren bij de eerste ssh. Als deze nog niet in het \textit{known hosts} bestand zit. Dit is het enige dat de EC2 role deed. Het wordt ook alleen aangeroepen als het IP adress gedefinieerd is.
\begin{lstlisting}
name: Add the instance to known hosts
local_action: command sh -c 'ssh-keyscan -t rsa {{ ec2_ip_address }}
                                             >> $HOME/.ssh/known_hosts'
when: ec2_$ip_address is defined
\end{lstlisting}

\newpage
\subsubsection{Common rol}
Voor de common rol werden we verwezen naar een andere artikel namelijk: \autocite{commonscottharney}. Deze rol was zeer uitgebreid, dus heeft het zijn eigen artikel.


\textbf{Sectie 1 common rol}

Git is zeer belangrijk in deze rol dus koos de auteur om dit als allereerst te installeren met de rol. De ssh sleutels worden ook gekopieerd van uit een private git repository. Erna wordt \textit{with\_items} gebruikt. Het is een looping optie in Ansible die hier gekozen git bestanden kopieerde. Het laatste deel zorgde ervoor dat de auteur zijn private bitbucket repo's beschikbaar zijn ,met zijn vorige sleutels, vanop de instance.
\begin{lstlisting}
- name: install git
become: yes
apt:
    name: git
    state: present

- name: send ssh id
copy:
    src: /home/sharney/.ssh/id_rsa
    dest: /home/sharney/.ssh/id_rsa
    mode: 0600

- name: git configfiles
become: no
copy: src={{ item.src }} dest={{ item.dest }}
with_items:
    - { src: 'dot_gitignore', dest: '/home/sharney/.gitignore' }
    - { src: 'dot_gitignore', dest: '/home/sharney/.gitignore' }
    - { src: 'dot_git', dest: '/home/sharney/.git' }

- name: bitbucket  key to known_hosts
known_hosts:
    key: "{{ lookup('pipe', 'ssh-keyscan -t rsa bitbucket.org')}}"
    name: bitbucket.org
    state: present
\end{lstlisting}
\newpage
\textbf{Sectie 2 common rol}

Dit deel toon \autocite{commonscottharney} zijn script, dat hij maakte om zijn private configuraties te doen. Dit werd ergens online gevonden, meerbepaald dankzij: \autocite{justinellin}.
\begin{lstlisting}
- name: do configfiles repo
become: no
script: /home/sharney/source/ansible-mystuff/configfiles_setup.sh  creates=/home/sharney/configfiles/.configfiles_done

- name: ssh id permissions fix 
become: no
file: path=/home/sharney/.ssh/id_rsa mode=0600
\end{lstlisting}

\begin{lstlisting}
#!/bin/sh

# do the bits to setup configflies repo
if [ ! -d $HOME/configfiles ]; then
mkdir $HOME/configfiles 
cd $HOME
git clone --no-checkout git@bitbucket.org:scott_harney/configfiles.git
git reset --hard origin/master
fi


if [ ! -e $HOME/Dropbox/.tmux.conf ]; then 
ln -s $HOME/Dropbox/.tmux.conf
fi
if [ ! -e $HOME/Dropbox/.dir_colors ]; then
ln -s $HOME/Dropbox/.dir_colors
fi

touch $HOME/configfiles/.configfiles_done
\end{lstlisting}

\newpage
\textbf{Sectie 3 common rol}

Het volgende deel van de rol voegde een ssh github sleutel toe. Voor het clonen van publieke repos later in het in de rol. Ook werden de andere nodige packages geïnstalleerd. Wel zijn sommige packages onnodig als er geen GUI aanwezig is op de instance.
\begin{lstlisting}
- name: github key to known_hosts
known_hosts:
    key: "{{ lookup('pipe', 'ssh-keyscan -t rsa github.com') }}"
    name: github.com
    state: present

- name: install packages for emacs and more
action: apt pkg={{ item }} state=installed install_recommends=yes
become: yes
with_items:
    - emacs24
    - emacs24-el
    - pandoc
    - tmux
    - zsh
    - ispell
    - vpnc
    - fonts-hack-ttf
    - ruby
    - ruby-aws-sdk
    - python-pip
    - python-pip-whl
    - virtualenv
    - curl
    - openjdk-8-jre
    - fonts-crosextra-caladea
    - fonts-crosextra-carlito 
\end{lstlisting}

\textbf{Sectie 4 common rol}

Dit waren gewoon python pip en ruby onderdelen die hielpen bij het beheren van AWS.
\begin{lstlisting}
- name: python pip install items
become: yes
pip: name={{ item }} state=present
with_items:
    - powerline-status
    - awscli
    - saws
- name: aws sdk v1 for ruby for awscli
become: yes
gem: name=aws-sdk-v1 state=present
\end{lstlisting}

\textbf{Sectie 5 common rol}

Dit deel zorgde voor de installatie van \textit{spacemacs}. \textit{spacemacs} is een linux tekst editor met de power en uitbreidbaarheid van \textit{Emacs} en de werking van \textit{vi/vim}.
\begin{lstlisting}
 name: check for spacemacs already imported
stat: path=/home/sharney/.emacs.d/spacemacs.mk
register: spacemacs_import

- name: mv .emacs.d out of the way for spacemacs if spacemacs not yet deployed
command: mv /home/sharney/.emacs.d /tmp
when: spacemacs_import.stat.exists == False

- name: spacemacs
git: repo=https://github.com/syl20bnr/spacemacs dest=/home/sharney/.emacs.d 
when: spacemacs_import.stat.exists == False

- name: for spacemacs org-protocol-capture-html
git: repo=https://github.com/alphapapa/org-protocol-capture-html.git dest=/home/sharney/source/org-protocol-capture-html

- name: fix .emacs.d/private via source copy
copy: src=/home/sharney/.emacs.d/private/private dest=/home/sharney/.emacs.d/private
when: spacemacs_import.stat.exists == False
# note: relying on the fact that the host running ansible has checked out private git repo
\end{lstlisting}

\textbf{Sectie 6 common rol}

Hier zette \autocite{commonscottharney} zijn tijdzone. Ook worden hier bestanden gesynchroniseerd die in zijn Dropbox stonden.
\begin{lstlisting}
- name: gen en_US.UTF-8 locale
become: yes
locale_gen: name=en_US.UTF-8 state=present

- name: set default locale to en_US.UTF-8
become: yes
command: /usr/sbin/update-locale LANG=en_US.UTF-8

- name: symlink for dir_colors
file: src=/home/sharney/Dropbox/.dir_colors path=/home/sharney/.dir_colors state=link

- name: symlink for .tmux.conf
file: src=/home/sharney/Dropbox/.tmux.conf path=/home/sharney/.tmux.conf state=link
\end{lstlisting}

\newpage
\textbf{Sectie 7 common rol}

Het laaste deel conifureerde de vpn. Ook worden de files van zijn website gepulled. Ten laatste is de customisatie van \textit{zsh}.
\begin{lstlisting}
- name: copy vpnc config
become: yes
copy:
    src: vpnc
    dest: /etc
    owner: root
    group: yes
    mode: 0700 

- name: git clone scottharney.com
git: repo=git@bitbucket.org:scott_harney/scottharney.com.git dest=/home/sharney/scottharney.com

- name: install ohmyzsh via git
git: repo=https://github.com/robbyrussell/oh-my-zsh.git  dest=/home/sharney/.oh-my-zsh depth=1 

- name: fix .zshrc
copy: src=/home/sharney/.zshrc dest=/home/sharney/.zshrc
\end{lstlisting}

\subsubsection{Ansible bestanden rol}
Deze rol pulled alle playbooks van \autocite{scottharney} zijn git repo \textit{ansible\_mystuff} en zet het op de server. Zo kan hij vanop deze instance aan zijn playbooks werken.
\begin{lstlisting}
- name: install ansible 
become: yes
apt:
    name: ansible
    state: present

- name: ansible-mystuff repo for deployment of util hosts
git: repo=git@bitbucket.org:scott_harney/ansible-mystuff.git  dest=~/source/ansible-mystuff    
\end{lstlisting}

\newpage
\subsubsection{OpenVPN rol}

Via deze kan \autocite{scottharney} zijn instance gebruiken op plaatsen die geen veilige verbinding zijn (bijvoorbeeld een café). Ook hij aan bestanden geraken van EC2 instances die geen internet gateway hebben. Deze instance wordt door deze rol een ssh host en een gateway naar \autocite{scottharney} zijn VPC. Omdat het een aparte rol is kunnen deze functie later hergebruikt worden.

\textbf{Sectie 1 OpenVPN rol}

Dit installeerde de openvpn package en kopieerde de al aanwezige configuraties naar de instance.
\begin{lstlisting}
- name: install openvpn
become: yes
apt:
    name: openvpn
    state: present

- name: copy openvpn configuration data
become: yes
copy: src=openvpn dest=/etc
\end{lstlisting}

\textbf{Sectie 2 OpenVPN rol}

Hier wordt ip forwarding aangezet en een \textit{iptables} is nu aanwezig voor het VPN verkeer.
\begin{lstlisting}
- name: set up ip forwarding
become: yes
sysctl: name="net.ipv4.ip_forward" value=1 sysctl_set=yes state=present reload=yes

- name: update /etc/rc.local
become: yes
copy: src=rc.local dest=/etc/rc.local mode=0755

- name: do iptables
become: yes
iptables: state=present table=nat chain=POSTROUTING source=192.168.3.0/24 out_interface=eth0 jump=MASQUERADE
\end{lstlisting}

\textbf{Sectie 3 OpenVPN rol}

Dit was volgens \autocite{scottharney} het moeilijste deel van zijn playbook. Voor dit deel heeft een lange tijd moeten troubleshooten. Het probleem dat hij ondervond was dat er een bug was die openvpn reports niet goed bij hield in Xenial. Zo heeft \autocite{scottharney} zijn bug kunnen oplossen.
\begin{lstlisting}
- name: fix debian bug in openvpn service https://bugs.launchpad.net/ubuntu/+source/openvpn/+bug/1580356?
become: yes
copy: src=openvpn@.service dest=/lib/systemd/system/
\end{lstlisting}

\textbf{Sectie 4 OpenVPN rol}

Er is ee een \textit{systemd} module in Ansible 2.2 maar \autocite{scottharney} koos er voor om dit toch via de command line te doen. Dit is iets dat hij later zou kunnen aanpassen naar de module van Ansible. Maar momenteel werkt dit goed genoeg
\begin{lstlisting}
- name: restart systemd daemon after updating unit config
become: yes
command: systemctl daemon-reload

- name: enable openvpn services
become: yes
command: systemctl enable openvpn.service

- name: restart openvpn services
become: yes
command: systemctl restart openvpn.service/
\end{lstlisting}











