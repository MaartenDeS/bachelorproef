\chapter{\IfLanguageName{dutch}{Stand van zaken}{State of the art}}
\label{ch:stand-van-zaken}

% Tip: Begin elk hoofdstuk met een paragraaf inleiding die beschrijft hoe
% dit hoofdstuk past binnen het geheel van de bachelorproef. Geef in het
% bijzonder aan wat de link is met het vorige en volgende hoofdstuk.

% Pas na deze inleidende paragraaf komt de eerste sectiehoofding.

%Dit hoofdstuk bevat je literatuurstudie. De inhoud gaat verder op de inleiding, maar zal het onderwerp van de bachelorproef *diepgaand* uitspitten. De bedoeling is dat de lezer na lezing van dit hoofdstuk helemaal op de hoogte is van de huidige stand van zaken (state-of-the-art) in het onderzoeksdomein. Iemand die niet vertrouwd is met het onderwerp, weet nu voldoende om de rest van het verhaal te kunnen volgen, zonder dat die er nog andere informatie moet over opzoeken \autocite{Pollefliet2011}.

%Je verwijst bij elke bewering die je doet, vakterm die je introduceert, enz. naar je bronnen. In \LaTeX{} kan dat met het commando \texttt{$\backslash${textcite\{\}}} of \texttt{$\backslash${autocite\{\}}}. Als argument van het commando geef je de ``sleutel'' van een ``record'' in een bibliografische databank in het Bib\LaTeX{}-formaat (een tekstbestand). Als je expliciet naar de auteur verwijst in de zin, gebruik je \texttt{$\backslash${}textcite\{\}}.
%Soms wil je de auteur niet expliciet vernoemen, dan gebruik je \texttt{$\backslash${}autocite\{\}}. In de volgende paragraaf een voorbeeld van elk.

%\textcite{Knuth1998} schreef een van de standaardwerken over sorteer- en zoekalgoritmen. Experten zijn het erover eens dat cloud computing een interessante opportuniteit vormen, zowel voor gebruikers als voor dienstverleners op vlak van informatietechnologie~\autocite{Creeger2009}.

In dit hoofdstuk wordt de stand van zaken besproken van cloud-init en Ansible. Eerst wordt uitgelegd wat Ansible is, wat de eigenschappen zijn en hoe het precies werkt. Daarna wordt er net hetzelfde met cloud-init gedaan. Ten laatste wordt er ook een literatuurstudie uitgevoerd op gevonden van het bachelorproefvoorstel.

\section{Ansible}
Ansible is een open source IT configuration management en deployment tool. Ansible hun grote doel is om besturingssysteem configuratie en de implementatie van software allemaal onder 1 systeem. De informatie werd gevonden met behulp van het document \textit{Ansible In Depth} \autocite{ansibleid}.

Ansible staat bekend als een systeem dat makkelijk te leren is als IT administrator, ontwikkelaar of manager. Het probeert er voor te zorgen dat het makkelijk te verstaan is en makkelijk om zelf op te bouwen. Zodat nieuwe gebruikers dit makkelijk kunnen oppikken. Ze proberen uniek te zijn door dingen door val aanpassing mogelijk te geven aan gebruikers voor de expert-gebruikers. Maar toch net zo toegankelijk voor de nieuwe gebruiker.

\subsection{Architectuur}
Eén van de belangrijkste verschillen tussen Ansible en andere configuratie management tools, is zijn architectuur. Ansible gaat uit van het ``push'' model. Ook is er geen additionele software nodig om machines bruikbaar te maken voor Ansible. Het heeft geen extra gebruikers of referenties nodig om te draaien. Het gebruikt gewoon de informatie die de user meegeeft. Daarbij hoort ook dat Ansible geen administrator of sudo toegang nodig heeft. Ansible wordt standaard bestuurt door een remote computer.

Dit zorgt ervoor dat Ansible veiliger wordt. Door alleen de informatie die de gebruiker meegeeft te gebruiken. Iemand die wel toegang heeft tot de server kan maar niet tot te remote computer kan geen aanpassingen pushen.

\subsection{Playbook - Roles}
Ansible voert de automatisatie en deployment uit via playbooks. Dit zijn yaml bestanden die beschrijven hoe de automatisatie moet verlopen. 

Deze playbooks bevatten verschillende ``plays'' die de automatisatie definiëren over verschillende hosts. Deze hosts staan bekend als de inventory. Elke ``play'' bevat verschillende taken die één, verschillende of alle hosts moeten uitvoeren. Elke taak roept een Ansible module aan, een klein stukje code dat een specifieke taak uitvoert. Deze kan kunnen zeer simpel zijn, een bestand op een machine zetten of een specifieke package installeren. Maar ze kunnen ook complex zijn zoals een gehele CloudFormation opstarten in Amazon EC2.
\begin{figure}[!htb]
    \center{\includegraphics[width=0.9\textwidth]{img/playbookex.png}}
    \caption{Voorbeeld van een Ansible playbook.}
    \label{fig:playbook}
\end{figure}

Ansible is geschreven zodat als ze de playbook uitvoeren ook checken of deze task nog moet gedaan worden. Bijvoorbeeld als een Ansible taak is om een webserver op te starten, zal Ansible deze alleen uitvoeren als de webserver nog niet is opgestart. Dit staat bekend als idempotente. Het zorgt ervoor dat de configuratie altijd snel en efficiënt wordt uitgevoerd.

Met Ansible kan je taken ook inkapselen in een role. Dit wordt gebruikt als je een specifieke configuratie meerdere keren moet uitvoeren, bijvoorbeeld het opzetten van een webserver. De Ansible Galaxy site bevat veel roles die kunnen gebruikt en aangepast worden voor het gebruik in een playbook.
\begin{figure}[!htb]
    \center{\includegraphics[width=0.9\textwidth]{img/agalaxy.png}}
    \caption{Ansible Galaxy site met lijst van roles.}
    \label{fig:agalaxy}
\end{figure}


\subsection{Omgevingen}
Ansible is even makkelijk te deployen in publieke of private cloud omgevingen, als in een lokale omgeving. Voor publieke of private cloud providers kan er gekozen worden voor: Amazon Web Services, Microsoft Azure, Rackspace,... Maar er kan ook op lokale infrastructuren gewerkt worden door middel van virtuele machines. Tools die hiervoor worden gebruikt zijn: VirtualBox, VMWare,...

\section{Cloud-Init}

TOEVEGEN BRONNEN + CHECKEN CLOUD SPELLING CLOUDINIT  ppt en site docs

Net als Ansible is cloud-init ook een soort van configuration manager en deployment tool. Het is geschreven in python. Cloud-init zorgt  voor de customisaties tijdens het opstarten van de cloud of virtuele instanties. Deze service gebeurt heel vroeg in het boot proces. De instantie zoekt naar de \textit{user data} (het cloud-init script) van de gebruiker en voert deze uit. 

De naam zegt het al zelf, maar cloud-init is een tool die meer wordt gebruikt door cloud instanties. Bijna alle cloud-providers hebben een functie waardoor je een cloud-init script kan meegeven.

\subsection{Modules}
Cloud-init heeft 6 hoofdpijlers waar het modules voor heeft, namelijk: schijf configuratie, commando's uitvoeren, gebruikers en groepen creeeren, beheren van packages, content bestanden schrijven en bootstrappen van Chef en/of Puppet. Er zijn nog andere modules maar dit zijn de 6 hoofdpijlers van cloud-init en daarbij de meeste gebruikte modules. Er kunnen ook zelf modules toevoegd worden door deze in Python te schrijven. Door middel van  de documentatie van BRONNENNNNNNNN wordt er per pijler uitleg gegeven. 

\subsubsection{Schijf configuratie}
Er is 1 module die wordt gebruikt voor de schijf configuratie, namelijk \textbf{Disk Setup}. Via deze module kan je simpele paritities en bestandsystemen configureren. Via de \textit{device\_aliases} richtlijnen kunnen er aliassen worden gemaakt voor de de block devices. Zodat er makkelijker naar deze kan worden verwezen. Via de \textit{disk\_setup} richtlijn wordt de partitie configuratie gedaan. De \textit{table\_type} richtlijn wordt gebruikt om de partitie tabel mee te gegeven. Ten laatste is er ook de \textit{fs\_setup} richtlijn deze wordt gebruikt om de systeembestand configuratie te doen.

\subsubsection{Commando's uitvoeren}
Voor het uitvoeren van commando's zijn er 2 modules: \textbf{Runcmd} en \textbf{Bootcmd}. Beide bevatten maar 1 richtlijn namelijk \textit{runcmd} en \textit{bootcmd}. Bij \textit{runcmd} worden de command's die worden meegegeven elke keer uitgevoerd als het script wordt gedraaid. Bij \textit{bootcmd} enkel alleen als de instantie wordt opgestart. Ook worden de commando's bij \textit{bootcmd} veel vroeger in het bootproces uitgevoerd.

\subsubsection{Gebruikers en groepen}
Ook voor Gebruikers en groepen is er slechts 1 module: \textbf{Users and Groups}.
\subsubsection{Packages}

\subsubsection{Content bestanden}

\subsubsection{Chef - Puppet }


\subsection{User Data - Cloud config}



