\chapter{\IfLanguageName{dutch}{Literatuurstudie}{State of the art}}
\label{ch:stand-van-zaken}

% Tip: Begin elk hoofdstuk met een paragraaf inleiding die beschrijft hoe
% dit hoofdstuk past binnen het geheel van de bachelorproef. Geef in het
% bijzonder aan wat de link is met het vorige en volgende hoofdstuk.

% Pas na deze inleidende paragraaf komt de eerste sectiehoofding.

%Dit hoofdstuk bevat je literatuurstudie. De inhoud gaat verder op de inleiding, maar zal het onderwerp van de bachelorproef *diepgaand* uitspitten. De bedoeling is dat de lezer na lezing van dit hoofdstuk helemaal op de hoogte is van de huidige stand van zaken (state-of-the-art) in het onderzoeksdomein. Iemand die niet vertrouwd is met het onderwerp, weet nu voldoende om de rest van het verhaal te kunnen volgen, zonder dat die er nog andere informatie moet over opzoeken \autocite{Pollefliet2011}.

%Je verwijst bij elke bewering die je doet, vakterm die je introduceert, enz. naar je bronnen. In \LaTeX{} kan dat met het commando \texttt{$\backslash${textcite\{\}}} of \texttt{$\backslash${autocite\{\}}}. Als argument van het commando geef je de ``sleutel'' van een ``record'' in een bibliografische databank in het Bib\LaTeX{}-formaat (een tekstbestand). Als je expliciet naar de auteur verwijst in de zin, gebruik je \texttt{$\backslash${}textcite\{\}}.
%Soms wil je de auteur niet expliciet vernoemen, dan gebruik je \texttt{$\backslash${}autocite\{\}}. In de volgende paragraaf een voorbeeld van elk.

%\textcite{Knuth1998} schreef een van de standaardwerken over sorteer- en zoekalgoritmen. Experten zijn het erover eens dat cloud computing een interessante opportuniteit vormen, zowel voor gebruikers als voor dienstverleners op vlak van informatietechnologie~\autocite{Creeger2009}.

In dit hoofdstuk wordt er een literatuur studie gedaan. Er zijn 3 artikels gevonden die worden besproken: \textit{An introduction to server provisioning with CloudInit}, \textit{Automated Ansible AWX Installation} en \textit{Using Ansible to Bootstrap My Work Environment Part 4}. De 3 artikels waren alle 3 ook een onderdeel van het bachelorproef voorstel.

\section{An introduction to server provisioning with CloudInit}
Het eerste artikel dat gevonden is heet: \textit{An introduction to server provisioning with CloudInit}. Het is geschreven door Viktor Petersson. \autocite{viktorpet} beschrijft in zijn artikel de basis van cloud-init en hoe een er mee kan worden gewerkt op CloudSigma. 

In deze literatuurstudie wordt er besproken wat de link met Ansible is volgens \autocite{viktorpet}. Een ander groot onderdeel van het artikel is hoe het Cloud config bestand wordt opgemaakt. Dat wordt hier ook besproken. Het deel waar er wordt getoond hoe je een CloudSigma server opstelt gaat niet worden besproken. Dit is een klein deel van het artikel en niet van nut voor deze bachelorproef.

\subsection{Verschil en samenwerking met Ansible volgens Viktor Petersson}
\autocite{viktorpet} wilt allereerst de lezer doen inzien dat cloud-init een specifieke plaats heeft in de server provisioning wereld. In \autocite{viktorpet} wordt er vermeld dat Cloud-init namelijk 1 eigenschap heeft, dat vele provisioning systemen zoals een Ansible, Puppet of Chef niet hebben. 

Terwijl cloud-init perfect kan gebruikt worden als een stand-alone provisioning systeem. Is het 1 van de weinig systemen die het gebruik met andere provisioning systeem ondersteunt, en zelf aanraadt volgens \autocite{viktorpet}. \autocite{viktorpet} vermeld ook dat hij prefereerdt cloud-init te gebruiken met Ansible.

\subsection{Setup cloudconfig bestand}
Het volgende onderdeel van \autocite{viktorpet} dat we bespreken is het opstellen van zijn cloud config bestand. In VERWIJZEN NAAR HOOFSTUK OPZOEKEN
