\chapter{\IfLanguageName{dutch}{Literatuurstudie}{State of the art}}
\label{ch:stand-van-zaken}


% Tip: Begin elk hoofdstuk met een paragraaf inleiding die beschrijft hoe
% dit hoofdstuk past binnen het geheel van de bachelorproef. Geef in het
% bijzonder aan wat de link is met het vorige en volgende hoofdstuk.

% Pas na deze inleidende paragraaf komt de eerste sectiehoofding.

%Dit hoofdstuk bevat je literatuurstudie. De inhoud gaat verder op de inleiding, maar zal het onderwerp van de bachelorproef *diepgaand* uitspitten. De bedoeling is dat de lezer na lezing van dit hoofdstuk helemaal op de hoogte is van de huidige stand van zaken (state-of-the-art) in het onderzoeksdomein. Iemand die niet vertrouwd is met het onderwerp, weet nu voldoende om de rest van het verhaal te kunnen volgen, zonder dat die er nog andere informatie moet over opzoeken \autocite{Pollefliet2011}.

%Je verwijst bij elke bewering die je doet, vakterm die je introduceert, enz. naar je bronnen. In \LaTeX{} kan dat met het commando \texttt{$\backslash${textcite\{\}}} of \texttt{$\backslash${autocite\{\}}}. Als argument van het commando geef je de ``sleutel'' van een ``record'' in een bibliografische databank in het Bib\LaTeX{}-formaat (een tekstbestand). Als je expliciet naar de auteur verwijst in de zin, gebruik je \texttt{$\backslash${}textcite\{\}}.
%Soms wil je de auteur niet expliciet vernoemen, dan gebruik je \texttt{$\backslash${}autocite\{\}}. In de volgende paragraaf een voorbeeld van elk.

%\textcite{Knuth1998} schreef een van de standaardwerken over sorteer- en zoekalgoritmen. Experten zijn het erover eens dat cloud computing een interessante opportuniteit vormen, zowel voor gebruikers als voor dienstverleners op vlak van informatietechnologie~\autocite{Creeger2009}.

In dit hoofdstuk wordt er een literatuur studie gedaan. Er zijn 2 artikels gevonden die worden besproken: \textbf{An introduction to server provisioning with CloudInit} en \textbf{Using Ansible to Bootstrap My Work Environment Part 4}. De 2 artikels waren alle 2 ook een onderdeel van het bachelorproef voorstel.

\section{An introduction to server provisioning with CloudInit}
Het eerste artikel dat gevonden is heet: \textbf{An introduction to server provisioning with CloudInit}. Het is geschreven door Viktor Petersson. \autocite{viktorpet} beschrijft in zijn artikel de basis van cloud-init en hoe een er mee kan worden gewerkt op CloudSigma. 

In deze literatuurstudie wordt er besproken wat de link met Ansible is volgens \autocite{viktorpet}. Een ander groot onderdeel van het artikel is hoe het Cloud config bestand wordt opgemaakt. En wat dit bestand heeft/geeft qua voordelen. Ook is er een deel waar er wordt getoond hoe er een CloudSigma server wordt opgesteld .Dit wordt worden besproken. Dit is een klein deel van het artikel en niet van nut voor deze bachelorproef/onderzoek.

\subsection{Verschil en samenwerking met Ansible volgens Viktor Petersson}
\autocite{viktorpet} wilt allereerst de lezer doen inzien dat cloud-init een specifieke plaats heeft in de server provisioning wereld. In \autocite{viktorpet} wordt er vermeld dat Cloud-init namelijk 1 eigenschap heeft, dat vele provisioning systemen zoals een Ansible, Puppet of Chef niet hebben. 

Terwijl cloud-init perfect kan gebruikt worden als een stand-alone provisioning systeem. Is het 1 van de weinig systemen die het gebruik met andere provisioning systeem ondersteunt, en zelf aanraadt volgens \autocite{viktorpet}. \autocite{viktorpet} vermeld ook dat hij prefereert cloud-init te gebruiken met Ansible.

\subsection{Setup cloud config bestand}
Het volgende onderdeel van \autocite{viktorpet} dat we bespreken is het opstellen van zijn cloud config bestand. Dat is het bestand waar de server configuraties wordt ingevoerd. \autocite{viktorpet} legt aan de hand van verschillende functies van het cloud config bestand uit wat de voordelen van cloud-init zijn. In Hoofdstuk \ref*{ch:cloud-init} is hier wel al meer over uitgelegd. Sommige delen gaan minder uitgebreid zijn omdat het al in Hoofdstuk \ref*{ch:cloudmodules} te vinden is.

\subsubsection{SSH}
Het eerste onderdeel van het cloud config bestand van \autocite{viktorpet} zijn de SSH keys. Cloud-init is volgens \autocite{viktorpet} handig om ssh keys toe te voegen aan de server. Zo kan er verbinding gemaakt worden met de server. 

\autocite{viktorpet} gebruikt 2 fictieve gebruikers en SSH Keys in zijn voorbeeld. Deze publieke SSH sleutels worden geïnstalleerd op de server voor de gekozen gebruiker. In het fictieve voorbeeld dus een een publieke sleutel voor user1 en user2 op de server host. Op Ubuntu Cloud Images als de gebruikers niet aanwezig zijn zullen deze worden geïnstalleerd op de standaard gebruiker ubuntu. 

\autocite{viktorpet} maakt verbinding met de server via het commando \textit{ssh ubuntu@IPADRESS}. Via de toegevoegde public keys gaat dit zonder problemen.
\begin{lstlisting}
#cloud-config
ssh_authorized_keys:
 - ssh-rsa AAA...user1@host
 - ssh-rsa AAA...user2@host
\end{lstlisting} 

\subsubsection{Systeem updates}
Het volgende voordeel van cloud-init volgens \autocite{viktorpet} is het uitvoeren van systeem updates tijdens de eerste boot. Dit is 1 van de functies die al werden besproken in Hoofdstuk \ref*{ch:cloudmodules}.

In zijn voorbeeld gebruikt \autocite{viktorpet} \textit{apt\_updgrade}. Terwijl er in Hoofstuk \ref*{ch:cloudmodules} werd gezien dat updates worden gedaan met \textit{package\_upgrade}. \textit{apt\_upgrade} is gewoon een alias voor \textit{package\_upgrade}.

Ook is \textit{apt\_update} zijn standaard waarde true als er packages worden geïnstalleerd op de server. 
\begin{lstlisting}
#cloud-config
apt_update: true
apt_upgrade: true
\end{lstlisting} 

\subsubsection{Installeren packages}
Het volgende dat wordt bespreken is het installeren van packages. Ook dit werd al besproken in Hoofdstuk \ref*{ch:cloudmodules}. Extra informatie die nog niet gekend was geeft \autocite{viktorpet} niet. Zijn voorbeeld staat wel ook hieronder
\begin{lstlisting}
#cloud-config
packages:
 - python-pip
 - fail2ban
 - vim
\end{lstlisting} 

\subsubsection{Hostname}
Wat \autocite{viktorpet} ook handig vindt aan cloud-init is het aanpassen van de hostname. Deze kan ook makkelijk worden aangepast.
\begin{lstlisting}
#cloud-config
hostname: mynode
fqdn: mynode.example.com
manage_etc_hosts: true
\end{lstlisting} 

\subsubsection{Commando's}
Als er sprake is van een meer geavanceerde gebruiker moeten er ook commando's worden uitgevoerd. Er zijn 2 opties voor commando's uit te voeren \textit{runcmd} en \textit{bootcmd}. Hierover kan er ook meer informatie worden gevonden in Hoofdstuk \ref*{ch:cloudmodules}.
\begin{lstlisting}
#cloud-config
runcmd:
 - ls -l /root
\end{lstlisting} 
\subsubsection{Server coniguration manager}
Een laatste voordeel volgens \autocite{viktorpet} is dat als meer geavanceerde gebruiker er ook de mogelijkheid is om een extra Server Configuaration Manager te gebruiken. 

Cloud-init ondersteunt samenwerking met onder andere Chef, Puppet en Salt.
\subsubsection{Alles samenbinden}
Ten laatste toon \autocite{viktorpet} hoe dit allemaal kan worden samen gegooid in bestand. Dit allemaal wordt samengevoegd in een YAMl bestand en bovenaan wordt er \textit{\#cloud-config} gezet. Dit werd ook vermeld in Hoofdstuk \ref*{ch:cloud-init}
\begin{lstlisting}
#cloud-config
ssh_authorized_keys:
- ssh-rsa AAA...user1@host
- ssh-rsa AAA...user2@host

hostname: mynode
fqdn: mynode.example.com
manage_etc_hosts: true

apt_update: true
apt_upgrade: true

packages:
- python-pip
- fail2ban
- vim

runcmd:
- ls -l /root
\end{lstlisting} 

\section{Using Ansible to Bootstrap My Work Environment Part 4}
Het tweede artikel is: \textbf{Using Ansible to Bootstrap My Work Environment Part 4}. Het is een Blogpost geschreven Scott Harney. \autocite{scottharney} beschrijft in zijn blogpost hoe hij zijn werk omgeving opstart via Ansible met behulp van cloud-init. 

% tot hier was spellcheck en woorden bekeken.

Het is het beste artikel dat werd gevonden waar Ansible en cloud-init beide expliciet werden gebruikt. Er werd voor het eerste een omgeving weergegeven die opgezet werd door een samenwerking van beide. In \autocite{scottharney} wordt beschreven hoe Scott Harney een EC2 instance op zet op AWS. AWS is 1 van de cloud providers die cloud-init ondersteunt. Het artikel is opgedeeld in 4 onderdelen. 

\textit{'Initial provisioning'}, in dit deel wordt de EC2 instance opgestart. Dit gebeurt met een eerste Ansible playbook dat ondere ook verwijst naar het cloud-init script dat nodig is. \textit{'Cloud-init'}, in het 2de gedeelte wordt het cloud-init script opgemaakt en uitgelegd wat alles doet en betekent. \textit{'Ansible folow up playbook'}, in dit gedeelte wordt de EC2 instance geconfigureerd met Ansible. \textit{'Launch and configure'} hier wordt de omgeving opgestart en worden sommige variablen nog ingevuld.

Het literatuur onderzoek gaat over de eerste 3 onderdelen gaan: initial provisioning, cloud-init en Ansible follow up playbook. Hierin worden Ansible en cloud-init besproken en geconfigureerd.

\subsection{Ansible initial playbook}
Er werd gekozen om de initiële provisioning en post-provisioning in 2 verschillende playbooks te doen. \autocite{scottharney} vond dit het handigste en vind het onnodig om deze tot 1 playbook te maken. 

Dit playbook zorgt voor het aanmaken en opstarten van de instance. Dit wordt gedaan via de Ansible role \textit{ec2}. Via deze role en dit playbook wordt ook het cloud config bestand meegegeven. \autocite{scottharney} gaf dit mee door middel van de optie \textit{user\_data}. 

\subsection{Cloud-init}
\autocite{scottharney} beschreef hier hoe hij de het cloud config bestand, waar hij in het eerste playbook naar verwees, heeft geconfigureerd. Een eerste opmerking van \autocite{scottharney} was dat cloud-init niet veel goede documentatie had. \autocite{scottharney} vond dit opmerkelijk omdat het een populaire tool is. Vervolgens beschreef \autocite{scottharney} hoe zijn cloud config bestand is opgedeeld.

\subsubsection{Begin}
Allereerst werd de normale gebruiker van \autocite{scottharney} aangemaakt met de nodige configuraties.

\begin{lstlisting}
#cloud-config
users:
- name: {{ ansible_user }}
ssh-authorized-keys:
- ssh-rsa  umm. nope
groups: [ 'admin', 'adm', 'dialout', 'sudo', 'lxd', 
		'plugdev', 'netdev' ]
shell: /bin/bash
sudo: ["ALL=(ALL) NOPASSWD:ALL"]
\end{lstlisting} 

\subsubsection{Server en host update}
In het 2de deel van het bestand nam \autocite{scottharney} zijn Hostname en fqdn en werden de packages upgedate. Ook update hij de instance zijn \textit{/etc/host} en update hij de tijdzone.

\begin{lstlisting}
hostname: "{{ item.hostname }}"
fqdn: "{{ item.fqdn }}"
manage_etc_hosts: true
timezone: US/Central
package_update: true
package_upgrade: true
\end{lstlisting} 

\subsubsection{Packages}
Om het laatste van zijn script uit te voeren had \autocite{scottharney} 2 packages nodig, namelijk: \textit{python} en \textit{awscli}. 

\begin{lstlisting}
packages:
- awscli
- python
\end{lstlisting}

\subsubsection{DNS zone bestand}
Het laatste deel was het belangrijkste deel van het cloud config bestand. \autocite{scottharney} moest zijn DNS zone bestand updaten. Zodat zijn EC2 instance correct werkte. Dit deed hij door met de optie \textit{write\_files} een script aan te maken om deze te updaten. Hij prefereerde deze methode dan het the pushen via \textit{systemd}.
\begin{lstlisting}
write_files:
- content: |
#!/bin/sh
FQDN=`hostname -f`
ZONE_ID="{{ zone_id }}"
TTL=300
SELF_META_URL="http://169.254.169.254/latest/meta-data"
PUBLIC_DNS=$(curl ${SELF_META_URL}/public-hostname 2>/dev/null)

cat << EOT > /tmp/aws_r53_batch.json
{
"Comment": "Assign AWS Public DNS as a CNAME of hostname",
"Changes": [
{
"Action": "UPSERT",
"ResourceRecordSet": {
"Name": "${FQDN}.",
"Type": "CNAME",
"TTL": ${TTL},
"ResourceRecords": [
{
"Value": "${PUBLIC_DNS}"
}]}}]}
EOT

aws route53 change-resource-record-sets --hosted-zone-id ${ZONE_ID} 
		--change-batch file:///tmp/aws_r53_batch.json
rm -f /tmp/aws_r53_batch.json
path: /var/lib/cloud/scripts/per-boot/set_route53_dns.sh
permissions: 0755e
\end{lstlisting}
\subsection{Ansible follow up playbook}
