\chapter{Verklarende woordenlijst}
verwijzen naar woordenlijst in inleiding. Ongekende termen worden gedefinieerd in de woordenlijst

ook nog verwijzen naar het voorstel enal in inleiding

\textbf{Server Configuration Management Tool}: Dit zijn tools die zorgen de voor automatisatie en configuratie van servers.

\textbf{Chef}: Chef is een server configuration management tool. Het zorgt voor de automatische configuratie op servers. Bij Ansible wordt er via playbooks gewerkt. Chef werkt met recipes, vandaar de naam.

\textbf{Puppet}: Puppet is ook een server configuration management tool. Samen met Ansible is dit de populairste.   

\textbf{Salt}: En ook Salt is een server configuration management tool.  

\textbf{Cloud Server Provider}: Dit is een bedrijf dat cloud computing aanbiedt. Dit kan zijn het aanmaken van servers op de cloud.

\textbf{Hetzner Cloud}: Hetzner Cloud is een cloud server provider.

\textbf{Vagrant}: Vagrant is een tool die virtuele machines aanmaakt en beheert. Vagrant werkt wel altijd samen met een ander programma dat deze dan draait.

\textbf{VirtualBox}: VirtualBox is een virtualisatie programma. Hiermee worden  virtuele machines gedraaid op een computer.

\textbf{Push}: 

\textbf{Pull}: 

\textbf{Deployment}: Software deployment zijn alle activiteiten ervoor zorgen dat een software systeem klaar is voor gebruik.

\textbf{Yaml bestand}: (Afkorting YAML Ain't Markup Language) Dit is een soort bestand dat vooral gebruikt wordt als configuratie bestand.

\textbf{Package}: In Linux worden alle applicaties in packages gestoken.

\textbf{CloudFormation}: Dit is een service dat helpt om Amazon Web Services op te zetten en te vormen.


\textbf{Amazon EC2}:

\textbf{Idempotentie}:

\textbf{Bootstrappen}:

\textbf{VPC}:

\textbf{Block devices}:

\textbf{Ubuntu}:

\textbf{CloudSigma}:

\textbf{SSH Keys}:

\textbf{ISO}:

\textbf{Vagrantfile}:

\textbf{Makefile}:

\textbf{Vagrant box}: