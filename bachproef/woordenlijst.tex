\chapter{Verklarende woordenlijst}
\textbf{Server Configuration Management Tool}: Dit zijn tools die zorgen de voor automatisatie en configuratie van servers.

\textbf{Chef}: Chef is een server configuration management tool. Het zorgt voor de automatische configuratie op servers. Bij Ansible wordt er via playbooks gewerkt. Chef werkt met 'recepten', vandaar de naam.

\textbf{Puppet}: Puppet is ook een server configuration management tool. Samen met Ansible is dit de populairste.   

\textbf{Salt}: Ook Salt is een server configuration management tool.  

\textbf{Cloud Server Provider}: Dit is een bedrijf dat cloud computing aanbiedt. Dit kan zijn het aanmaken van servers op de cloud.

\textbf{Hetzner Cloud}: Hetzner Cloud is een cloud server provider.

\textbf{Vagrant}: Vagrant is een tool die virtuele machines aanmaakt en beheert. Vagrant werkt wel altijd samen met een ander programma dat deze dan draait.

\textbf{VirtualBox}: VirtualBox is een virtualisatie programma. Hiermee worden  virtuele machines gedraaid op een computer.

\textbf{Pull}: Pull betekent aanvragen van een programma of computer.

\textbf{Push}: Push is het tegenovergestelde van Pull. Push stuurt data naar een programma zonder eerst een aanvraag te doen.

\textbf{Deployment}: Software deployment zijn alle activiteiten ervoor zorgen dat een software systeem klaar is voor gebruik.

\textbf{Yaml bestand}: (YAML Ain't Markup Language) Dit is een soort bestand dat vooral gebruikt wordt als configuratie bestand.

\textbf{Package}: In Linux worden alle applicaties in packages gestoken.

\textbf{CloudFormation}: Dit is een service dat helpt om Amazon Web Services op te zetten en te vormen.

\textbf{Amazon EC2}: Dit is een webservice van Amazon voor de Cloud Servers te beheren en configureren.

\textbf{AWS}: (Amazon Web Services) Dit is een Server Cloud Provider van Amazon.

\textbf{Idempotentie}: Een programma heeft idempotentie als, het programma niet meer verandert wanneer de operatie nog is wordt uitgevoerd.

\textbf{Bootstrappen}: Het laden van een besturingssysteem.

\textbf{VPC}: (Virtual Private Cloud) Een private cloud oplossing in een publieke cloud infrastructuur.

\textbf{Block devices}: Dit is een opslag apparaat dat bestanden leest en schrijft in 'fixed-size' blocks.	

\textbf{Ubuntu}: Een Linux besturingssysteem.

\textbf{CloudSigma}: Dit is een Zwitserse server cloud provider.

\textbf{SSH Keys}: Manier om toegang te verkrijgen tot een server. Via een publieke sleutel, heb je toegang tot een systeem die bijpassende private sleutel heeft.

\textbf{FQDN}: (Fully Qualified Domain Name) Dit het volledige domein adres.

\textbf{Git}: Git is een gratis opensource distributiesysteem. Het is een makkelijk tool waar projecten op kunnen worden bewaard of worden gedeeld met anderen.

\textbf{Repository (repo)}: Dit is een map of bestandslocatie waar alle projectbestanden staan.

\textbf{BitBucket}: Bitbucket is een repository hosting systeem dat git ondersteunt. Het wordt vooral gebruikt door bedrijven.

\textbf{Vagrantfile}: Het configuratie bestand voor het aanmaken van een virtuele machine met Vagrant.

\textbf{Vagrant box}:Box's zijn het package formaat voor vagrant omgevingen.

\textbf{Makefile}: het configuratie bestand om met het commando \textit{make} een programma te bouwen.

\textbf{VPN}: (Virtual Private Network) Dit creëert een veilige geëncrypteerde connectie over een minder veilige netwerk, meestal een publiek netwerk.

\textbf{Python}: Python it is een programmeertaal.

\textbf{Pip}: Pip is een package beheerder, die packages geschreven in python installeert.

\textbf{Ruby}: Ruby is een programmeertaal

\textbf{Base64}: Base64 is een tool om binaire data te coderen of decoderen naar een tekst formaat.
