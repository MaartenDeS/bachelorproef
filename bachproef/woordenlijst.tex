\chapter{Verklarende woordenlijst}
verwijzen naar woordenlijst in inleiding. Ongekende termen worden gedefinieerd in de woordenlijst

ook nog verwijzen naar het voorstel enal in inleiding

\textbf{Chef}: Chef is een server configuration management tool. Het zorgt voor de automatische configuratie op servers.  

\textbf{Puppet}: Puppet is ook een server configuration management tool. Samen met Ansible is dit de populairste.   

\textbf{Salt}: En ook Salt een server configuration management tool. 

\textbf{Configuration Management Tool}:

\textbf{Cloud Server Provider}:

\textbf{Hetzner Cloud}:


\textbf{Vagrant}:

\textbf{VirtualBox}:

\textbf{Push}:

\textbf{Deployment}: Software deployment zijn alle activiteiten ervoor zorgen dat een software systeem klaar is voor gebruik.

\textbf{Yaml bestand}: (Afkorting YAML Ain't Markup Language) UITLEG. Het wordt vooral gebruikt als configuratie bestand.

\textbf{Package}:

\textbf{CloudFormation}:

\textbf{Amazon EC2}:

\textbf{idempotentie}:

\textbf{Bootstrappen}:

\textbf{VPC}:

\textbf{Block devices}:

\textbf{Ubuntu}:

\textbf{CloudSigma}:

\textbf{SSH Keys}:

\textbf{ISO}:

\textbf{Vagrantfile}:

\textbf{Makefile}:

\textbf{Vagrant box}: