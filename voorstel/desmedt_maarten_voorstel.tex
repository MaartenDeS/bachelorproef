%==============================================================================
% Sjabloon onderzoeksvoorstel bachelorproef
%==============================================================================
% Gebaseerd op LaTeX-sjabloon ‘Stylish Article’ (zie voorstel.cls)
% Auteur: Jens Buysse, Bert Van Vreckem

\documentclass[fleqn,10pt]{voorstel}

%------------------------------------------------------------------------------
% Metadata over het voorstel
%------------------------------------------------------------------------------

\JournalInfo{HoGent Bedrijf en Organisatie}
\Archive{Bachelorproef 2018 - 2019} % Of: Onderzoekstechnieken

%---------- Titel & auteur ----------------------------------------------------

% TODO: geef werktitel van je eigen voorstel op
\PaperTitle{Onderzoek naar samenwerking tussen Cloud-Init en Ansible}
\PaperType{Onderzoeksvoorstel Bachelorproef} % Type document

% TODO: vul je eigen naam in als auteur, geef ook je emailadres mee!
\Authors{Maarten De Smedt\textsuperscript{1}} % Authors
\CoPromotor{Simon Lepla\textsuperscript{2} (Be-Mobile)}
\affiliation{\textbf{Contact:}
  \textsuperscript{1} \href{mailto:maarten.desmedt.v5415@student.hogent.be}{maarten.desmedt.v5415@student.hogent.be};
  \textsuperscript{2} \href{mailto:simon.lepla@be-mobile.com}{simon.lepla@be-mobile.com};
}

%---------- Abstract ----------------------------------------------------------

\Abstract{Bij Be-Mobile wordt voor de installatie van servers al jaren provisioning systemen gebruikt zoals Ansible en Cloud-init, te samen of apart. Ansible maakt gebruik van hun playbooks om de server juist te configureren. Cloud-Init doet dat door middel van hun cloud-images. Maar waar moet je overstappen naar de ander, hoe gebruik je ze het best samen en maken ze mekaar misschien overbodig? Allemaal vragen die nog onbeantwoord zijn. Aangezien een bedrijf als Be-Mobile en ook gewone gebruikers gebruik maken van beide is het belangrijk dat daar een duidelijk antwoord op wordt gevonden. In dit document kan u verder lezen over: wat deze 2 technologieën precies zijn en waar ze verschillen en gelijk zijn, 5 verschillende literatuur studies/onderzoeken over dit onderwerp , hoe ik dit wil onderzoeken, de verwachte resultaten en conclusies. Ik verwacht niet dat onderzoek één resultaat zal hebben maar eerder verschillende resultaten voor verschillende type omgevingen. De verwachte conclusie zal dus eerder zijn waar je welke het beste gebruikt en waar je best gaat combineren. In de toekomst zal er door dit onderzoek hopelijk meer duidelijkheid zijn tussen de samenwerking(of misschien is die er niet) van Cloud-init en Ansible.}
	
	%Hier schrijf je de samenvatting van je voorstel, als een doorlopende tekst van één paragraaf. Wat hier zeker in moet vermeld worden: \textbf{Context} (Waarom is dit werk belangrijk?); \textbf{Nood} (Waarom moet dit onderzocht worden?); \textbf{Taak} (Wat ga je (ongeveer) doen?); \textbf{Object} (Wat staat in dit document geschreven?); \textbf{Resultaat} (Wat verwacht je van je onderzoek?); \textbf{Conclusie} (Wat verwacht je van van de conclusies?); \textbf{Perspectief} (Wat zegt de toekomst voor dit werk?).





%---------- Onderzoeksdomein en sleutelwoorden --------------------------------
% TODO: Sleutelwoorden:
%
% Het eerste sleutelwoord beschrijft het onderzoeksdomein. Je kan kiezen uit
% deze lijst:
%
% - Mobiele applicatieontwikkeling
% - Webapplicatieontwikkeling
% - Applicatieontwikkeling (andere)
% - Systeembeheer
% - Netwerkbeheer
% - Mainframe
% - E-business
% - Databanken en big data
% - Machineleertechnieken en kunstmatige intelligentie
% - Andere (specifieer)
%
% De andere sleutelwoorden zijn vrij te kiezen

\Keywords{Systeembeheer --- Cloud-Init --- Ansible} % Keywords
\newcommand{\keywordname}{Sleutelwoorden} % Defines the keywords heading name

%---------- Titel, inhoud -----------------------------------------------------

\begin{document}

\flushbottom % Makes all text pages the same height
\maketitle % Print the title and abstract box
\tableofcontents % Print the contents section
\thispagestyle{empty} % Removes page numbering from the first page

%------------------------------------------------------------------------------
% Hoofdtekst
%------------------------------------------------------------------------------

% De hoofdtekst van het voorstel zit in een apart bestand, zodat het makkelijk
% kan opgenomen worden in de bijlagen van de bachelorproef zelf.
%---------- Inleiding ---------------------------------------------------------

\section{Introductie} % The \section*{} command stops section numbering
\label{sec:introductie}

Voor de installatie van servers wordt al jaren Ansible gebruikt. Ansible is een universele 'taal' die taken voor servers automatiseert door middel van hun playbook. Zo een Ansible playbook is een georganiseerde unie van scripts dat het werk voor de server configuratie definieert.

Cloud-Init is momenteel een van de industrie standaarden voor het opbouwen van cloud servers, het maakt gebruik van cloud images. Dat zijn besturingssysteem sjablonen en elke instantie begint als een identieke kloon van elke andere instantie. De gebruikersgegevens geven elke cloud instantie haar persoonlijkheid. Doormiddel van cloud-int worden deze gegevens op de instantie toegepast. 

Dit zijn allebei provisioning systemen. Op een verschillende manier doen ze in theorie hetzelfde. Ze brengen de server allebei naar de gewenste toestand van de gebruiker. Doordat deze op verschillende manieren werken, wordt er in de praktijk in combinatie met beide gewerkt. In dat geval gebruik je eerste cloud-init om de server naar een gewenste toestand te brengen waar na Ansible het kan overnemen.

Maar is dit de perfecte samenwerking? Het bedrijf Be-mobile is opzoek naar het antwoord. In dit onderzoek bestuderen we waar deze mekaar aanvullen en hoe ze dan op een perfect performante manier werken. Natuurlijk is het goed mogelijk dat deze mekaar niet aanvullen en dan is het onderzoek waar en wanneer je voor wat moet kiezen en waarom dit mekaar overbodig maakt. 

In dit onderzoek zullen we dit trachten te ontdekken. De vragen waar het, de antwoorden op wil vinden zijn:
\begin{itemize}
	\item Vullen Ansible en Cloud-Init mekaar aan, of maken ze elkaar overbodig?
	\item Ook hoe ze mekaar aanvullen en/of hoe ze mekaar overbodig maken?
\end{itemize}

%Hier introduceer je werk. Je hoeft hier nog niet te technisch te gaan.

%Je beschrijft zeker:

%\begin{itemize}
%  \item de probleemstelling en context
%  \item de motivatie en relevantie voor het onderzoek
%  \item de doelstelling en onderzoeksvraag/-vragen
%\end{itemize}

%---------- Stand van zaken ---------------------------------------------------

\section{Stand Van Zaken}
\label{sec:state-of-the-art}

Ansible en Cloud-Init zijn geen onbekende voor mekaar dus er is wel een basis gevonden waarop er verder kan gewerkt worden. Maar zijn niet zo bekend dat de ultieme samenwerking al gevonden is. In het onderzoeken van de 2 systemen zijn er 5 artikels gevonden die een samenwerking beschrijven of afraden. 


In de artikels \textbf{"An introduction to server provisioning with CloudInit"\autocite{Cloudsigma}}, \textbf{“Using Ansible to Bootstrap My Work Environment Part 4” \autocite{scottharney}} en \textbf{“Customizing Cloud Assembly Deployments with Cloud-Init” \autocite{vmware}} is er een vaak voorkomend fenomeen gevonden. Meestal gebruiken de gebruikers/auteurs eerst cloud-init om de machine op te starten, daarna ansible voor de verdere specifieke eigenschappen.


Het artikel geschreven door \textcite{Cloudsigma} is wel interessant doordat hij ook andere systemen buiten cloud-init heeft getest, waaronder puppet en chef. Maar toch verkiest hij om Ansible te gebruiken. Dit betekent dat een samenwerking met ansible optimaler is.


De artikels \textbf{“Zero Touch Provisioning of Infoblox Grid on OpenStack using Ansible”\autocite{infoblox}} en \textbf{"Automated Ansible AWX Installation" \autocite{deven}} beschrijven dan weer hoe beide elke als een apart systeem kunnen worden gebruikt zonder de hulp van de ander. Maar toch zijn er een paar gelijkenissen zo gebruikt \textcite{deven} in "Automated Ansible AWX Installation" notities van een Ansible installatie om het met cloud-init te installeren.


Er is duidelijk al bekend dat er samengewerkt kan worden, maar ook soms weer niet. Maar criteria om te bepalen wanneer welke het best is, is nog niet bekend. 








%Hier beschrijf je de \emph{state-of-the-art} rondom je gekozen onderzoeksdomein. Dit kan bijvoorbeeld een literatuurstudie zijn. Je mag de titel van deze sectie ook aanpassen (literatuurstudie, stand van zaken, enz.). Zijn er al gelijkaardige onderzoeken gevoerd? Wat concluderen ze? Wat is het verschil met jouw onderzoek? Wat is de relevantie met jouw onderzoek?

%Verwijs bij elke introductie van een term of bewering over het domein naar de vakliteratuur, bijvoorbeeld~\autocite{Doll1954}! Denk zeker goed na welke werken je refereert en waarom.

% Voor literatuurverwijzingen zijn er twee belangrijke commando's:
% \autocite{KEY} => (Auteur, jaartal) Gebruik dit als de naam van de auteur
%   geen onderdeel is van de zin.
% \textcite{KEY} => Auteur (jaartal)  Gebruik dit als de auteursnaam wel een
%   functie heeft in de zin (bv. ``Uit onderzoek door Doll & Hill (1954) bleek
%   ...'')


%---------- Methodologie ------------------------------------------------------
\section{Methodologie}
\label{sec:methodologie}

Dit onderzoek zal gevoerd worden door virtuele server omgevingen op te zetten met Cloud-init en/of Ansible. De programma’s die hierbij zullen worden gebruikt zijn VirtualBox, Vagrant en Atom. 

Virtualbox is een programma om virtuele machines op te draaien. Met Vagrant kan je van in je shell met een simpel commando virtuele machines op starten. Atom is dan weer een teksteditor om de configuratie goed in neer te pennen. Normaal is deze het vermelden niet waar maar door de overzichtelijke manier van werken die het aanbiedt is deze toch veel beter dan een gewone Notepad.

Met Ansible zal er voor verschillende servers een testomgeving worden opgezet. Dit zal dan ook gebeuren met cloud-init. Deze resultaten en omgevingen kunnen dan worden  vergeleken met elkaar. De technische eigenschappen/resultaten worden dan onder de loep genomen: de performantie, de snelheid van het opstarten,… Ook zal er worden gekeken naar de config files van beiden en kan er worden gekeken welke daar per omgeving “beter” is. M.a.w. welke er op een kortere overzichtelijkere manier hetgeen kan bekomen. Be-Mobile kan ook nog verdere criteria bepalen waardoor we een keuze gaan maken. Daarna worden ook  combinaties van de twee aangemaakt en dan kunnen we deze vergelijken met de originele servers. 

Ook kan het zijn dat er vragen worden gesteld aan medewerkers van het bedrijf Be-Mobile om te bekijken wat hun mening over beide is.

%Hier beschrijf je hoe je van plan bent het onderzoek te voeren. Welke onderzoekstechniek ga je toepassen om elk van je onderzoeksvragen te beantwoorden? Gebruik je hiervoor experimenten, vragenlijsten, simulaties? Je beschrijft ook al welke tools je denkt hiervoor te gebruiken of te ontwikkelen.

%---------- Verwachte resultaten ----------------------------------------------
\section{Verwachte resultaten}
\label{sec:verwachte_resultaten}
De verwachtingen zijn dat een combinatie van cloud-init en Ansible meestal het beste zal zijn. Dat er voor elke server tot een bepaald moment cloud-init of ansible  zal worden gebruikt waarna ansible of cloud-init het zal overnemen. Ook zijn de verwachtingen dat er voor elke omgeving een andere oplossing zal zijn. Er zal veel afhangen van het type besturingssysteem dat draait en van de servers om te kunnen bepalen welke het best wordt gekozen. 

De verwachtingen zijn ook dat niet bij alles een combinatie zal worden gebruikt. Ook kan het misschien zijn dat het beter is dat je één van beide kiest voor een bepaalde server omgeving.


%Hier beschrijf je welke resultaten je verwacht. Als je metingen en simulaties uitvoert, kan je hier al mock-ups maken van de grafieken samen met de verwachte conclusies. Benoem zeker al je assen en de stukken van de grafiek die je gaat gebruiken. Dit zorgt ervoor dat je concreet weet hoe je je data gaat moeten structureren.

%---------- Verwachte conclusies ----------------------------------------------
\section{Verwachte conclusies}
\label{sec:verwachte_conclusies}
De verwachte conclusie is dat elke server omgeving een andere uitkomst zal hebben. Er veel zal afhangen van welke server je precies wilt draaien op welk systeem. Dan kan er worden gekozen voor een bepaalde combinatie van beide of misschien één van beide. De conclusie zal niet zijn welke van de twee beter is. Maar eerder hoe ze het best gebruikt worden per serveromgeving.



%Hier beschrijf je wat je verwacht uit je onderzoek, met de motivatie waarom. Het is \textbf{niet} erg indien uit je onderzoek andere resultaten en conclusies vloeien dan dat je hier beschrijft: het is dan juist interessant om te onderzoeken waarom jouw hypothesen niet overeenkomen met de resultaten.



%------------------------------------------------------------------------------
% Referentielijst
%------------------------------------------------------------------------------
% TODO: de gerefereerde werken moeten in BibTeX-bestand ``voorstel.bib''
% voorkomen. Gebruik JabRef om je bibliografie bij te houden en vergeet niet
% om compatibiliteit met Biber/BibLaTeX aan te zetten (File > Switch to
% BibLaTeX mode)

\phantomsection
\printbibliography[heading=bibintoc]

\end{document}
